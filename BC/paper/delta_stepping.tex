\section{Delta stepping}
\label{sec:delta_stepping}
%
Let us consider the case of computing BC for weighted graphs; we assume for 
the rest of this discussion that the weights are all non-negative.
%
In a weighted graph, to compute shortest paths, we replace the breadth-first 
search (BFS) kernel with a variant of Dijkstra's algorithm~\cite{dijkstra59}.
%
The \textit{variant} we use is called ``delta-stepping'' and was first proposed
by Meyer and Sanders~\cite{meyer-diss} in 2003.
%
Delta-stepping algorithm allows for more parallelism as it is label-correcting
rather than label-setting; label-setting algorithms settle the distance of 
each vertex from the source vertex only once.
%
For example, Dijkstra's algorithm picks the vertex with the shortest path at 
each step and relaxes the rest of the distances based on paths that go through
this vertex; inherently, this is a sequential algorithm as there is always a 
single vertex that has the shortest path to the source vertex.
%
Delta-stepping algorithm, shown in the Algorithm~\ref{alg:delta_stepping}, 
allows for reinsertions of vertices even after their distances have been 
settled and thereby increases parallelism.
